%%report template for pattern recognition SS2016

\documentclass[english, paper=a4]{scrartcl}
\usepackage[utf8]{inputenc}
% images
\usepackage{graphicx}
%math
\usepackage{amsmath,amssymb}
%code
\usepackage{algorithm}
\usepackage[noend]{algpseudocode}
\makeatletter
\def\BState{\State\hskip-\ALG@thistlm}
\makeatother

\usepackage{subcaption}
\captionsetup{compatibility=false}
\usepackage{multirow}
\usepackage{color}
\usepackage{enumitem}
\usepackage{float}

\begin{document}

\graphicspath{{images/}}


%%------------------------------------------------------
%% provide your input here:
\title{Baysian Learning with Cuda} 

\subtitle{GPU Architecture} 

\author{Marc Haubenstock (1525175)}



%%------------------------------------------------------

\maketitle

\section{Hidden Markov Model(HMM)}

\subsection{Overview}

\begin{figure}[H]
\centering

\includegraphics[scale=0.4]{"symbols"}
  \caption{The notation of a HMM \cite{cuhmm}}
\end{figure}


\subsection{Forward}

\begin{figure}[H]

\centering
\includegraphics[scale=0.4]{"FW"}
 \caption{Pesudo-Code for the Forward Algorithm \cite{cuhmm}}

\end{figure}

The algorithm produces a data structure called a trellis, denoted by \(\alpha\) of size \(N \times T \) for a sequence of T observations of size N. The trellis can be viewed as a matrix where each cell \(\alpha(t,i)\) holds the probability of being in state \(i\) while having seen the observations until \(t\). The complexity of the algorithm is due to the for loops. For \(M\) observation sequences the complexity of the algorithm is 
$\mathcal{O}(MN^2T)$. Due to the dependency on a previous value as shown line 7, the outer most for loop can not be parallised. Instead, we can exploit the fact that all our \(M\) observation sequences are independent, this can be computed in parallel, reduced the complexity to $\mathcal{O}(cMT)$, where c is the execution time of the parallel code \cite{cuhmm}.
\subsection{Viterbi}

\subsection{Baum-Welch}


\begin{figure}[H]

\centering
\includegraphics[scale=0.4]{"BW"}
 \caption{Pesudo-Code for the Backward Forward Algorithm \cite{cuhmm}}

\end{figure}

The general structure of the Baum-Welch algorithm is similar to its forward counterpart. Most notably instead of iterating from the start of the observation sequence, the algorithm starts at the last observation, moving backwards through the trellis. The parallisation can be achieved in a similar manner to the forward algorithm, but due to the change in indexing for the trellis \(\alpha\) in line 11 this can not be repesented as matrix multiplcation \cite{cuhmm}.

The idea of this algorithm is to (randomly) initialise the A and B matrices, perform the forward, then the backward pass on them. Finally with the computed values we can re-estimate the matrices and repeat the process.

Sadly numerous errors in the pseudo code made the implementation more difficult than presented. Specifically in line 9, the probability \(p\) is produced from the transition matrix \(A\) not the trellis, as is evident from the inconsistent indexing at \(\alpha\). Moreover, at line 10. the \(\beta\) values is not dependent on \(i-1\) but rather \(i\) \cite{hmm}. 

\begin{figure}[H]

\centering
\includegraphics[scale=0.4]{"estimation"}
 \caption{Estimation of the A and B Matricies \cite{cuhmm}}

\end{figure}

Similarly the estimated values of the emission matrix B is not dependent on \(\epsilon(j)\), but should be on \(\gamma\), as it is defined in literature \cite{hmm}. 

\newpage


\section{Implementation}

\subsection{Overview}

As seen in the previous section it is not possible to fully parallize the individual algorithms due to their dependence on a value of a previous time \(t\). However, hmms are usually computed with many observation sequences \textbf{M}. Each algorithm acts on one such sequence. The main idea presented in \cite{cuhmm} exploits this by parallizing over \textbf{M} instead of \(T\). 

\begin{figure}[H]
\centering

\includegraphics[scale=0.3]{"3d_trellis"}
  \caption{Graphical representation of the data structure presented in \textit{A Cuda Implementation of Hidden Markov Models}\cite{cuhmm}}
\end{figure}


Thus, the for-loop of the T domain remains in the code while the CUDA kernel computes all M slices at time t in parallel. Graphically this can be seen as moving through the 3D data structure either front-to-back, in the forward case, or back-to-front in the backward case. The current element is being computed by applying the following formulae.

\begin{center}

\( D_{ij} = B_{O_i} .* C_i \times A_j \)

\end{center}

Where \( B_{O_i}\) is the row of the emission matrix at observation \(O_i\), \(C_i\) is the previous slice and \(A_j\) is the j\textsuperscript{th} column of the transition. matrix A.

\begin{figure}[H]
\centering

\includegraphics[scale=0.3]{"slice"}
  \caption{The computation of a slice\cite{cuhmm}}
\end{figure}

Finally a stipulation of the outlined method is that:

\begin{itemize}
\item The number of states and sequences must be a multiple of block size 16
\item The number of output sequences must be of the same length.
\end{itemize}

\subsection{Forward}
\begin{verbatim}
	initTrellis<<<M,N>>>(..);

	for(int t = 0; t < T; t++){
		...
		
		 // copute B_o
		computeBRow<<<M,N>>>
		(M_noOfObsSequences, V_noOfObsSymbols, T_noOfObservations, 
		N_noOfStates, dev_O_obsSequence_2D, dev_B_obsEmissionProbs_2D, i, dev_B);
		
		cudaDeviceSynchronize();
		
		 // W = B_o .* C_i
		pointwiseMul<<<M,N>>>
		(dev_W, dev_B, &dev_3D_Trellis[(i - 1) * M_noOfObsSequences * N_noOfStates]);
		
		cudaDeviceSynchronize();
		
				// wrapper function for cublas multiply; D = W x A
		cublasDeviceMultiply(M_noOfObsSequences, N_noOfStates, N_noOfStates, dev_W, 
		dev_A_stateTransProbs_2D, &dev_3D_Trellis[i * M_noOfObsSequences * N_noOfStates]); 

		
		cudaDeviceSynchronize();
		
		...
		
	
	}
	
	double *last_slice 
	= &dev_3D_Trellis[(T_noOfObservations - 1) * M_noOfObsSequences * N_noOfStates];	
	
	// compute likelihood
	for (int i = 0; i < M_noOfObsSequences; i++)
	{

		int smBytes = 64 * sizeof(double);
		int grid = N_noOfStates / 64;
		reduce_1 <<< grid, 64, smBytes >>>(&last_slice[i*N_noOfStates], dev_A_odata);

		memcpyVector(host_A_odata, dev_A_odata, N_noOfStates, cudaMemcpyDeviceToHost);

		host_likelihoods_1D[i] = host_A_odata[0] + host_A_odata[1];

	}

\end{verbatim}

The implementation follows the pseudo-code outlined by Liang. First we compute the slices of our 3D data structure by looping over T. Instead of only computing a single value as outlined by the eqution, we compute all values for a slice at once.Once we computed all values, we compute the likelihood for each observation sequence. No pseudo-code was given, however, by looking at literature we know we have to compute this as a summation of all terms for a single observation sequence \cite{hmm}. We implementing this summation as a reduction in \(N\). The reduction uses the simple interleaved model and is outlines as the first method in \textit{Optimizing Parallel Reduction in CUDA} \cite{reduction}.

\subsection{Viterbi}

\subsection{Baum-Welch}

\begin{verbatim}


	initMatrix<<<N_noOfStates,V_noOfObsSymbols>>>(dev_B_prime_3D,M_noOfObsSequences);
	initMatrix<<<N_noOfStates,N_noOfStates>>>(dev_A_prime_3D, M_noOfObsSequences);
	
	// overwrite the "true" transfer and emission matricies with the random ones
	copyMatrix(dev_A_prime_3D, host_A_stateTransProbs_2D,
	 N_noOfStates, N_noOfStates, M_noOfObsSequences);
	copyMatrix(dev_B_prime_3D, host_B_obsEmissionProbs_2D, 
	N_noOfStates, V_noOfObsSymbols, M_noOfObsSequences);
	
	initBeta<<<M_noOfObsSequences, N_noOfStates>>>(dev_beta_3D, V_noOfObsSymbols);
	
	ForwardAlgorithmSet(host_Pi_startProbs_1D, host_A_stateTransProbs_2D, 
	host_B_obsEmissionProbs_2D,host_O_obsSequences_2D, N_noOfStates, 
	V_noOfObsSymbols, T_noOfObservations, M_noOfObsSequences,host_likelihoods_1D...);

	for (int t = T_noOfObservations - 1; t >= 0; t--) {
	
	UpdateGammaGPU << <M_noOfObsSequences, N_noOfStates >> >
	(dev_gamma_3D, dev_beta_3D, dev_3D_Trellis_Alpha, t, dev_likelihood, V_noOfObsSymbols);

		cudaDeviceSynchronize();

		UpdateEpsilonReductionErrorGPU <<<M_noOfObsSequences, N_noOfStates >>>
		(epsilon_reduction_grid_error, dev_beta_3D, dev_3D_Trellis_Alpha, t, dev_likelihood);
		
		if (t > 0){

			// Computes the B matrix in term D = B .* C x A
			ComputeBDevice << < M_noOfObsSequences, N_noOfStates >> >
			(M_noOfObsSequences, V_noOfObsSymbols, T_noOfObservations, 
			N_noOfStates, dev_O_obsSequences_2D, dev_B_obsEmissionProbs_2D, t, dev_B);
			cudaDeviceSynchronize();

			// All dimensions are multipe of Warp Sizes
			// Compute W =  B .* C i.e. beta(t,i) * b_it
			pointwiseMatrixMul << <M_noOfObsSequences, N_noOfStates >> >
			(dev_W, dev_B, &dev_beta_3D[t * M_noOfObsSequences * N_noOfStates]);

			cudaDeviceSynchronize();

			// Compute D = W x A, D = beta(t,i)*p
			cublasMultiplyDouble(M_noOfObsSequences, 
			N_noOfStates, N_noOfStates, dev_W, dev_A_stateTransProbs_2D, dev_D);

			cudaDeviceSynchronize();

			updateBeta << <M_noOfObsSequences, N_noOfStates >> >
			(dev_beta_3D, dev_D, t, V_noOfObsSymbols);

			cudaDeviceSynchronize();


			for (int j = 0; j < N_noOfStates; j++){
				UpdateEpsilonGPU <<<M_noOfObsSequences, N_noOfStates >>>
				(dev_epsilon_3D, dev_beta_3D,
				dev_3D_Trellis_Alpha, t, dev_likelihood, j, dev_D);

				cudaStatus = cudaDeviceSynchronize();
			} // end of for j

		} // end of if
	
	} // end of for t
	
	// Estimate Matricies - 	// sum up all values and reductions then divide

	ColumReduction_Height <<<1, N_noOfStates >>>
	(epsilon_reduction_grid_error, M_noOfObsSequences);
	ColumReduction_Height <<<N_noOfStates, N_noOfStates >>>
	(dev_epsilon_3D, M_noOfObsSequences);

	ColumReduction_Height <<< N_noOfStates,V_noOfObsSymbols>>>
	(dev_gamma_3D, M_noOfObsSequences);
	ColumReductionGamma_Depth <<<1, N_noOfStates >>>
	(dev_gamma_3D, 0, V_noOfObsSymbols,M_noOfObsSequences, gamma_reduction_grid);

	EstimateA <<<N_noOfStates, N_noOfStates>>>
	(dev_A_prime_3D, dev_epsilon_3D, epsilon_reduction_grid, 
	epsilon_reduction_grid_error, 0, M_noOfObsSequences);

	EstimateB <<<V_noOfObsSymbols, N_noOfStates>>>
	(dev_B_prime_3D, dev_gamma_3D, gamma_reduction_grid, 0, M_noOfObsSequences);
\end{verbatim}

The values presented in the pseudo code (Figure 3) at lines 6 and 7 can be computed easily as CUDA kernels. Similarly the values of \(\beta\) follow a similar procedure as the \(\alpha\) values in the forward algorithm, and can be computed accordingly. The \(\epsilon\) values on the other hand depend on 3 variables \(i,j,m\). Because of this, we opted for a simple approach and iterate over \(N\), to compute this value, increasing the complexity to $\mathcal{O}(cMNT)$.

Since the formula for computing the estimate \(\hat{b}\) is incorrect we had to adopt our own approach.

\begin{figure}

\centering

\( \hat{b}_j(v_k) = \frac{\text{expected number of times in state j observing symbol } v_k}{\text{expected number of times in state j}} \)
\caption{Definition of b hat taken form \cite{hmm}}

\end{figure}

This is implemented as the reduction concerning the gamma data structures in our code. First, we compute all values of \(\gamma_t(i,j)\) by summing over \(M\), then we compute \(\gamma_t(j)\) by summing over the symbol domain \(V\), and store this in our reduction data structure.

\section{Evaluation}

\subsection{Data Generation}

\subsection{Comparison}

\subsection{Conclusion}



%%------------------------------------------------------



\bibliographystyle{plain}
\bibliography{HMM}
%% References can be stored in a seperate bib-file (see lit.bib). References, that are cited in the report using \cite are automatically added to the reference list. For more information: http://www.bibtex.org/Using/
%%------------------------------------------------------
\end{document}
